\section{Layer 2 vs. Layer 3 circuits}

There is one major difference between the circuits provided by DYNES and the circuits provided by NSI that poses a serious problem for our use of circuits in PhEDEx. DYNES provides layer-3 circuits, i.e. circuits at the IP level. NSI only provides layer-2 circuits, at the ethernet level.

This is a problem because PhEDEx is high-level middleware, and knows only about source and destination hostnames and port numbers. It knows nothing about low-level details of the network such as ethernet addresses or anything to do with the topology of the network, it merely calls a transfer-layer tool (such as FTS\cite{FTS} or FDT\cite{FDT}) to handle the actual transfer. These tools take their input in the form of a SURL (a 'Storage URL') which specifies the host, port, protocol and local pathname on that host to access the file. PhEDEx is able to switch between different layer-3, IP-based circuits by changing the hostname (or IP address) that it uses in the SURL, the rest happens transparently.

Because NSI doesn't provide layer-3 circuits, this mechanism cannot work, and we need another approach.